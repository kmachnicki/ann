\chapter{Wstęp}
\section{Cele i założenia projektowe}
Celem projektu było zapoznanie się z tematem uczenia maszynowego, a konkretnie z sieciami neuronowymi.
Realizacja projektu miała umożliwić zrozumienie zasady działania oraz porównanie wydajności sieci neuronowych używających propagacji wstecznej oraz wariantu o nazwie \textit{Extreme Learning Machines,} na podstawie analizy danych z badań cytologicznych.
\section{Wstęp teoretyczny}
Rozwój uczenia maszynowego w ostatnich latach jest bardzo ważny z punktu widzenia wielu dziedzin, jednak szczególnie ważny dla medycyny.

Komputery stają się szybsze i są w stanie przetwarzać większe ilości danych, co przekłada się na możliwość użycia ich do analizowania np. wyników badań.
Dzięki rozwojowi techniki, mogą one pomagać lekarzom w diagnozowaniu, co pozwala na szybsze decyzje na temat leczenia pacjenta.

\textit{Sieci neuronowe} stanowią jedno z możliwych podejść do tematu uczenia maszynowego.
Są one modelowane na podobieństwo neuronów w mózgu, a zastosowanie znajdują między innymi w problemach klasyfikacji -- np. klasyfikacji stopnia złośliwości raka, tak jak to miało miejsce w projekcie.

Wzrost popularności zawdzięczają pojawieniu się szybszych komputerów oraz ogromnej ilości danych dostępnych do uczenia ich, co pozwoliło na zredukowanie wpływu głównych wad sieci neuronowych -- wolnego uczenia się oraz wymagania dużych zestawów danych uczących.
\subsection{Budowa sieci neuronowych}
Prosta sieć neuronowa może być zbudowana z trzech warstw neuronów:
\begin{itemize}
	\item{warstwy neuronów wejściowych,}
	\item{warstwy neuronów ukrytych,}
	\item{warstwy neuronów wyjściowych.}
\end{itemize}
Każda z warstw odbiera dane, a następnie po obliczeniu wartości funkcji aktywacji, przesyła je do kolejnej warstwy.
\begin{figure}[h!]
	\centering
	\includegraphics[width=0.4\linewidth]{img/model.png}
	\label{Rysunek}
	\caption{Schemat prostej sieci neuronowej. Źródło: wikipedia.org.}
\end{figure}