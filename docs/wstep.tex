\chapter{Wstęp}

\section{Cele i założenia projektowe}

Celem projektu było zapoznanie się z tematem uczenia maszynowego, a konkretnie z sieciami neuronowymi.
Realizacja projektu miała umożliwić zrozumienie zasady działania oraz porównanie wydajności sieci neuronowych używających propagacji wstecznej oraz wariantu o nazwie \textit{Extreme Learning Machines}, na podstawie analizy danych pacjentów.

\section{Wstęp teoretyczny}

Rozwój uczenia maszynowego w ostatnich latach jest bardzo ważny z punktu widzenia wielu dziedzin, jednak szczególnie ważny dla medycyny.

Komputery stają się szybsze i są w stanie przetwarzać większe ilości danych, co przekłada się na możliwość użycia ich do analizowania np. wyników badań.
Dzięki rozwojowi techniki, mogą one pomagać lekarzom w diagnozowaniu, co pozwala na szybsze decyzje na temat leczenia pacjenta.

\textit{Sieci neuronowe} stanowią jedno z możliwych podejść do tematu uczenia maszynowego.
Są one modelowane na podobieństwo neuronów w mózgu, a zastosowanie znajdują między innymi w problemach klasyfikacji -- np. klasyfikacji stopnia złośliwości raka, tak jak to miało miejsce w projekcie.
Wzrost popularności zawdzięczają pojawieniu się szybszych komputerów oraz ogromnej ilości danych dostępnych do uczenia ich, co pozwoliło na zredukowanie wpływu głównych wad sieci neuronowych -- wolnego uczenia się oraz wymagania dużych zestawów danych uczących.

Problem badany podczas projektu jest to problem klasyfikacji komórek rakowych. Jest on trudny, ponieważ jest problemem, który \textbf{nie jest zbalansowany} -- przypadki nowotworu złośliwego stanowią mniejszą część badanych. Wpływa to znacząco na proces uczenia się klasyfikatora i jakość klasyfikacji.

Klasyfikator miał za zadanie nauczyć się rozpoznawać jedną z dwóch klas -- G2 oraz G3 -- z której każda odpowiada odpowiednio 6 i 7 oraz 8 i 9 punktom w systemie Scarffa-Blooma-Richardsona.
