\chapter{Aplikacja}

\section{Wykorzystane technologie}

Aplikacja została stworzona przy wykorzystaniu języka \texttt{Python} w wersji \texttt{3.5}.

Biblioteka \texttt{scikit-learn} zawiera wiele narzędzi przydatnych w tematyce uczenia maszynowego, dlatego została wykorzystana w projekcie.
Użyto biblioteki w wersji deweloperskiej -- \texttt{0.18.dev0}\footnote{http://scikit-learn.org/dev/documentation.html} -- ze względu na dostępność wymaganych narzędzi, m.in. funkcji służących do
\begin{itemize}
	\item{zbudowania klasyfikatora opartego o sieci neuronowe,}
	\item{przeprowadzenia walidacji krzyżowej,}
	\item{stworzenia rankingu cech,}
	\item{stworzenia macierzy pomyłek.}
\end{itemize}

Ponieważ \texttt{scikit-learn} nie posiada zaimplementowanych \textit{Extreme Learning Machines,} w projekcie wykorzystano moduł \texttt{Python-ELM}\footnote{https://github.com/dclambert/Python-ELM}. 
Wyznaczony on był na oficjalnej stronie \textit{ELM}\footnote{http://www.ntu.edu.sg/home/egbhuang/elm\_codes.html} jako jedna z bazowych implementacji.
Udostępniony kod nie był przystosowany do działania pod wersją \textit{Pythona 3}, wymagał więc poprawek.

Wszystkie wykorzystane moduły i ich przeznaczenie widnieją w tabeli \ref{tab:usedmodules}.

\begin{table}[h!]
    \centering
    \caption{Wykorzystane moduły.}
    \label{tab:usedmodules}
    \begin{tabular}{p{3cm}p{2cm}p{11cm}}
        \toprule
        \textbf{Nazwa} & \textbf{Wersja} & \textbf{Opis} \\
        \midrule
        \texttt{scikit-learn} & 0.18 & Sieć neuronowa uczona metodą wstecznej propagacji błędu, selekcja cech, walidacja krzyżowa, macierz błędu. \\
        \texttt{Python-ELM} & 0.3 & \textit{Extreme Learning Machines}. \\
        \texttt{numpy} & 1.11.0 & Obliczenia numeryczne. \\
        \texttt{matplotlib} & 1.5.1 & Tworzenie wykresów. \\
        \bottomrule
    \end{tabular}
\end{table}

\newpage

\section{Budowa}

Na aplikację składa się kilka modułów:

\begin{itemize}
    \item \texttt{main.py} - główny moduł uruchamiający badania.
    \item \texttt{algorithms.py} - uruchamia eksperymenty.
    \item \texttt{dataset.py} - importuje dane z pliku CSV i przechowuje je w wygodnej dla programisty postaci.
    \item \texttt{grapher.py} - tworzy wykresy z wyników badań.
    \item \texttt{helper.py} - zawiera klasy pomocnicze do przechowywania danych.
    \item \texttt{consts.py} - zawiera ustawienia badań i parametry dla algorytmów.
\end{itemize}

Funkcje oraz klasy z modułu \texttt{sklearn}, z których skorzystano w projekcie: \texttt{neural\_network.MLPClassifier}, \texttt{feature\_selection.SelectKBest}, \texttt{model\_selection.StratifiedKFold}, \texttt{metrics.confusion\_matrix}.
\section{Wdrożenie}

W celu przygotowania aplikacji do działania, należy dodać katalog projektu do zmiennej środowiskowej \texttt{PYTHONPATH}. Można to zrobić za pomocą polecenia:

\begin{verbatim}
$ export PYTHONPATH="${PYTHONPATH}:${PWD}"
\end{verbatim}

Kolejnym krokiem jest uruchomienie skryptu instalacyjnego \texttt{install.sh}, który zainstaluje wymagane pakiety z pliku \texttt{requirements.txt} oraz pobierze poprawiony moduł \texttt{Python-ELM} do katalogu \texttt{modules}:

\begin{verbatim}
./install.sh
\end{verbatim}

Teraz można uruchomić aplikację poprzez wywołanie skryptu głównego \texttt{main.py}:

\begin{verbatim}
python3 main.py
\end{verbatim}

Działanie aplikacji powinno zakończyć się wygenerowaniem wykresów.